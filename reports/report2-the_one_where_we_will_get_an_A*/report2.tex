\documentclass[a4paper,12pt]{article}
\usepackage{fullpage}
\usepackage{hyperref}
\usepackage{url}
\usepackage{graphicx}
\usepackage{polski}
\usepackage[utf8]{inputenc}

\setlength{\parindent}{0pt}
\addtolength{\parskip}{\baselineskip}

\title{3rd Year Group Project\\Report Two\\\emph{Progress \& Revisions}}

\author{
    \small{Rafał Szymański}\\
  	\and
    \small{Maciek Albin}\\
    \and
    \small{Sam Wong}\\
    \and
    \small{Suhaib Sarmad}\\
		\and
		\small{Jamal Khan}\\
		\and
		\small{\{rs2909, mja108, sw2309, sss308, jzk09\}@doc.ic.ac.uk}
		\and
		\\Department of Computing - Imperial College London
}

\date{}

\begin{document} 
	\maketitle
	
	\section{Report I Amendments}
  After submitting our first report we realised the Iteration plan we proposed was flawed. In many parts instead of focusing on the user facing features we have reported on the technical aspects of what we will do. One whole iteration was dedicated to ``Debugging, refactoring and optimisation''. That is not in the spirit of Agile Programming and, we believe, may incline the team to focus on less important things. With this in mind we've decided it would be very advisable if we revised our iteration plan rather quickly and followed that one, instead of the one submitted in Report One.

  We've also realised the product and feature descriptions given in the first report suffered from similar misconceptions. We all knew what we wanted to build, but when tasked with putting our thoughts on paper we again focused on less relevant technical details, rather then overarching user facing features.

  This section will detail the new iteration plan and describe what the project goals are in a more feature oriented manner. Bare in mind this is the iteration timeline and feature list we've agreed on soon after submitting the first report and there have been some changes to this detailed in the \emph{Progress} section. 

  \subsection{Project Description and Key Requirements}
  In this project we want to create a website that will allow people to see an aggregated public Twitter reaction to top news stories emerging around the world. We want people to be able to glance at our UI, se what's happening right now and what are the dominant trends on Twitter related to that story.

  The key requirements of our project are:
  \begin{itemize}
   \item Displaying top news stories from Google News
   \item Displaying relevant most influential tweets next to the stories
   \item Displaying aggregated wordclouds next to the stories. This means analysing which words appear most often with relation to the news story.
   \item The UI needs to be fluid and updating in real-time and on its own - the user should not be forced to hit a refresh button at all. It should also be fun to look at.
  \end{itemize}

  The extensions we want to implement are:
  \begin{itemize}
   \item Attaching pictures related to the news story. These would be aggregated from Twitter and news articles.
   \item Attaching a general category to news stories (art, politics, etc.). This would be based on the tweets related to the story, not the story itself. We want to know what people think the news is about, not what news editors tell them.
   \item Displaying the general approval of the story. We can measure whether a tweet is positive or negative towards the topic (this is called \emph{sentiment analysis}) and we want to use these measurements to tell users whether the reaction to the news story is positive or negative.
  \end{itemize}

  \subsection{Iteration Details and Timeline}
  This iteration timeline starts at the week of our initial report. Previous weeks were utilised to decide on the goals of our project and coordinate with our project supervisor.
  \begin{description}
   \item[Iteration 1] \emph{Weeks 1-2}\\
   This iteration will also require us to do all of the initial setup. Find a way to fetch tweets relevant to news stories, etc. Hence the longer time period required.
   \begin{itemize}
     \item First attempt at the UI. It will be polished and new features will be added throughout iterations.
     \item Display current news stories.
     \item Generate keywords from news stories and fetch relevant tweets.
   \end{itemize}
   \item[Iteration 2] \emph{Week 3}
   \begin{itemize}
     \item Wordclouds generated from tweets related to news stories added to UI.
     \item Top tweets related to news stories added to UI.
   \end{itemize}
   \item[Iteration 3] \emph{Week 4}
   \begin{itemize}
     \item Add the ability to analyse whether the tweets are positive or negative and present the aggregated result in the UI.
     \item Fetch pictures relevant to news stories and present them in the UI.
   \end{itemize}
   \item[Iteration 4] \emph{Week 5}
   \begin{itemize}
     \item Categorise tweets and based on these categories categorise the news stories themselves. Add that to UI.
   \end{itemize}
   \item[Iteration 5] \emph{Week 6}
   \begin{itemize}
     \item Final application polish and validating if everything is \textbf{really really} working before final submission.
   \end{itemize}
  \end{description}
	
	\section{Progress}
	We are at the end of week three of the iteration schedule. We are more or less on schedule but our iteration plan needed some changes. We'll talk about the revised iteration plan in the next section, for now let's focus on what we've accomplished and what problems we have had.
	
	\subsection{Iteration 1}
	We have completed this iteration, but some parts of it were finished long after the other parts. To start with we had to do all of initial setup (continuos integration, basic database, deciding on and writing the basic architecture), which took us about a week to complete and we consider that part a success.
	
	One of the decisions we've made was to have the analysis part and the UI part of the application as separate as possible. We still think it was a good decision (it is going to make extending the application much easier), but it created a bit of a disconnect between the team responsible for the backend and the frontend team. We believe that this caused us to put off connecting the UI to the API and the API to the database for too long. It ended up being a part of the second iteration and there were some unforeseen problems that meant we had to put off implementing top tweets in the UI for later. The UI and the analysis were good on their own, but connecting them together proved to be more difficult. In the future we plan to diminish the results of this disconnect by properly utilising continuos integration (so far we haven't been able to see all the changes live, but it should change soon) and shifting people between the UI and the analysis teams.
	
	The other problem we've encountered was of a technical nature. We tried to design our own keyword generator, but it turned out to be challenging and started to take quite a lot of time. Since generating keywords was not a goal of our project we decided to use an external API\footnote{\url{http://www.alchemyapi.com/}} to perform that task. It works, but unfortunately the quality of keywords is quite low (we get keywords such as \emph{video}, or \emph{news} which make for a lot of irrelevant tweets). We provisionally added improving the keyword generation as part of one of the next iterations.
	
	\subsection{Iteration 2}
	Most of the iteration has been completed. Wordclouds are being generated and displayed in the UI, and we have the backend part done for displaying top tweets. Unfortunately for reasons outlined in the previous section we have not been able to implement the UI part responsible for displaying them. Part of the integration between the UI and the backend analysis thread has been moved here and has been completed.
	
	\section{Revisions}
	
	Our key requirements have not changed, but rather we have made a few extensions that have been mentioned in the first section of this report. The main ones include having pictures for each news story and also seeing the general public reaction to the particular news story by using sentiment analysis. We have discussed the proposed changes with our supervisor, who has approved.
  
  There have been no infrastructure changes to our project. We are still using a Python backend, with a JavaScript (jQuery) front end, so we can still use our computers for the testing stage. Having said this, we have also set up continuous integration on a VPS where we push code.
  
  We initially though that our progress would be measured by getting visible results. As our project is web based these results would be seeing news stories on a web page and relevant tweets. However our view on this has changed, we have been writing a lot of code and testing its functionality. Although we do value that we need to have a good looking and working web app, we are also focussing on making all the backend functionality perfect so that we do not face any problems in the next few week as we pursue making a good looking interface.
	
	
	
	\section{People Management}
	
	As it often happens with group projects, problems relating group interactions and dynamics surface. Firstly, in our group, everyone knows each other well and is good friends with everyone else, which reduces the authoritative attitude of any one group member towards others. In a real life project, you have a manager with an authoritative position to do things such as lay you off, which certainly induces at least a little bit of motivation. In a university project where all the members are good friends, it is harder to establish an authority that is able to motivate others, especially during a time where everyone has a lot of other personal tasks at hand, motivation has to come from within the group members.
	
	During this term all of us have lot of other personal responsibilities, including a lot of coursework for other subjects, setting up and studying for internship interviews for next year, and teaching PMT classes, among many others. Considering every one of us has so many personal tasks that directly affect just him and not the group, we are seeing decreased motivation towards the project from all group members. We are directly experiencing \emph{Social Loafing}\footnote{\url{http://en.wikipedia.org/wiki/Social_loafing}}, which is ``the phenomenon of people exerting less effort to achieve a goal when they work in a group than when they work alone.'' It still seems that we are having problems keeping to the iteration task assignment on Trello\footnote{\url{http://trello.com}}, ie we are not completing the iteration subtasks within the timeline. The root cause of this is mostly likely related to the above explained \emph{social loathing}, and diffusion of responsibility within a group without a clear authoritative figure.
	
	Our current proposed solution is rationalising to everyone the relative importance of this project compared to all other coursework - $440/1700$ points. Receiving a \textbf{C} for our first report was disappointing and a blow to our motivation, as none of us had gotten a C in Imperial before, but on the other hand it was a sign we need increase the quality of our work and approach the project more seriously.
	
	We have separated our group of five people as follows: 2 for UI, and 3 for backend, so everyone is doing what they are more comfortable with. Both of these subteams do pair programming. Many times one of us knows something useful, for example a specific MongoDB query syntax, and can help the other without having to resolve to Google. This ensures a good flow of information between the team, and levels out our knowledge. Nevertheless, what we still thrive to have is a \emph{bus count}\footnote{\url{bit.ly/j1zquA} - quite interesting article.}\footnote{Bus count - ``how many people in your team have to get hit by a bus before you’re all dead in the water''} of the size of our group - we want everyone to understand, and if needed, to work on any part of our stack.
	
	We try to establish fair contribution by everyone by assigning to everyone tasks on Trello. We have planned to have frequent scrum meetings regarding the completion of this tasks, but instead, since we all see each other during lectures anyways, we usually discuss the progress without having a dedicated meeting. What we need to do is actually get into the habit of having a dedicated progress meeting more frequently, where everyone has to showcase what they achieved in the previous 2 or 3 days.
	
	\section{Ethical and Environmental Impact}	
	
	The ethical issues we have to address are as follows:
	
	\begin{itemize}
	  
	  \item All technologies that we are using are open source. Python has an OSI-approved license\footnote{\url {http://docs.python.org/license.html}}, our server is written using Flask which is BSD Licenced\footnote{\url{http://flask.pocoo.org/docs/license/}} and MongoDB is also open source\footnote{\url{http://www.mongodb.org/display/DOCS/Licensing}}.
	  
	  \item We will eventually want to store Twitter usernames and passwords of our users, one way of doing this is storing the password in our database, which is risky. If our database is infiltrated then we will be compromising our users Twitter credentials. A way to get over this is to make an OAuth Twitter App and in this way the password remains a credential only shared between the user and Twitter and not us. 
	  
  \end{itemize}
  

\end{document}