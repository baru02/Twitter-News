\documentclass{report}
\usepackage{fullpage}
\usepackage{amssymb}
\usepackage[utf8]{inputenc}
\usepackage{url}

\setlength{\parindent}{0pt}
\addtolength{\parskip}{\baselineskip}

\title{Twitter News Generation - Final Report}

\author{
    \small{Rafał Szymański}\\
  	\and
    \small{Maciek Albin}\\
    \and
    \small{Sam Wong}\\
    \and  
    \small{Suhaib Sarmad}\\
		\and
		\small{Jamal Khan}\\
		\and
		\small{\{rs2909, mja108, sw2309, sss308, jzk09\}@doc.ic.ac.uk}
		\and
		\\Department of Computing - Imperial College London
}



\begin{document} 
	\maketitle
	\tableofcontents
	\newpage

	\section{High Level Overview}
	
	\section{Technical Overview}
	
	\section{Software Engineering Issues}
	
	
		\subsection{People and Team Management}

		We deem it important to discuss how we have managed our team and worked as a team. As it often happens with group projects, problems relating to group interactions and dynamics surface. Firstly, in our group, everyone knows each other well and is good friends with everyone else, which reduces the authoritative attitude of any one group member towards others. In a real life project, you have a manager with an authoritative position to do things such as lay you off, which certainly induces at least a little bit of motivation. In a university project where all the members are good friends, it is harder to establish an authority that is able to motivate others, especially during a time where everyone has a lot of other personal tasks at hand; motivation has to come from within the group members.

		During this term all of us had lot of other personal responsibilities, including a lot of coursework for other subjects, setting up and studying for internship interviews for next year, and teaching PMT classes, among many others. Considering every one of us has so many personal tasks that directly affect just him and not the group, we saw decreased motivation towards the project from all group members. We have experienced \emph{Social Loafing}\footnote{\url{http://en.wikipedia.org/wiki/Social_loafing}}, which is ``the phenomenon of people exerting less effort to achieve a goal when they work in a group than when they work alone.'' We have had problems keeping to the iteration task assignment on Trello\footnote{\url{http://trello.com}}, ie were not, as individuals, completing the iteration subtasks within the timeline. The root cause of this was likely related to the above explained \emph{social loathing}, and diffusion of responsibility within a group without a clear authoritative figure.

		Our proposed solution was rationalising to all group members the relative importance of this project compared to all other coursework - $440/1700$ points. Receiving a \textbf{C} for our first report was disappointing and a blow to our motivation, as none of us had gotten a C in Imperial before, but on the other hand it was a sign we need increase the quality of our work and approach the project more seriously. This, combined with a meeting with Robert Chatley to discuss problems with report number 1, ensured grades of A+, and A, for reports 2 and 3 respectively. 
		
		Even though over the course of the project, we, as a team, have improved in terms of teamwork, there is still a lot to be improved. We see three important features that we lacked, but that would characterize a more efficient team. Firstly, it is necessary to have an authoritative figure which will watch over the group and assign tasks. It is important that task assignment is followed by the team members. Of course, all topics should be open to a reasonable debate, but in the event progress stalls, the authoritative figure is there to take a decision and lead the project forward. Secondly, it would be beneficial if, when working on the project, the team members looked at each other as co-workers and not simply as school friends. If everyone is good friends outside of the project, it is necessary to introduce a change in attitude, so that, while remaining friends, team members can work as co-workers towards a common goal - the completion of the project. Without a change in attitude, we risk the teamwork project to be a considered a fun 'game' of not much importance, and the impact of this will be visible on the final output. Thirdly, as said by Peter Drucker\footnote{\emph{"What gets measured, gets managed"}}, we claim that is vital to measure progress frequently. Trello was a big help in establishing who is to do what, but what was lacking was an enforcement of these assignments. We have planned to have frequent scrum meetings regarding the completion of tasks, but instead, since we all saw each other during lectures anyways, we usually discussed the progress without having had a dedicated meeting. This was an unsatisfactory decision - we should have gotten into the habit of having dedicated progress meetings, where everyone has to showcase what they achieved in the previous 2 or 3 days. Under the knowledge that one has to have something to present at a team meeting, there is a much higher chance that said team member will have produced an amount of work worth discussing. Otherwise, with no set meetings, there is less incentive for individuals to contribute fairly and frequently.

		We now discuss the assignment of tasks and group subdivision. We have separated our group of five people as follows: 2 for UI, and 3 for backend, so everyone is doing what they are more comfortable with. From time to time, but maybe not as often as we could have done, both of these subteams utilized pair programming. Many times one of us knew something useful, for example a specific MongoDB query syntax, and can help the other without having to resolve to Google. This ensures a good flow of information between the team, and levels out our knowledge. A visible problem was the following scenario: we are working in the same location, but not currently pair programming. Person $A$ is concentrating and has a train of thought, while person $B$, interrupts to ask for a triviality such as the aforementioned MongoDB query syntax, which one easily find on google within 30 seconds. Now $A$ has lost concentration and needs to get back on their train of thought. We see that $B$ has shortened the amount of work for himself, but has lowered the total productivity and output of the group. This is to be avoided at all costs. There is a time to discuss and ask questions, and there is a time to work quietly. The group most not get itself into a state of perpetual back-and-forth questioning of, for example, API syntax, or else group productivity is greatly diminished. Nevertheless, when we set to do a session of pair programming, output and learning were both increased.
		
		 We discussed that the group was separated into frontend and backend, but we still thrived to have a \emph{bus count}\footnote{\url{bit.ly/j1zquA} - quite interesting article.}\footnote{Bus count - ``how many people in your team have to get hit by a bus before you’re all dead in the water''} of the size of our group - we wanted everyone to understand, and if needed, to work on any part of our stack. In the end, this has not fully materialized, as the frontend group don't know the intricacies involved in the threading of the backend processes, and the backend team is not aware of the various UI hacks the frontend team had to develop to have a nice cross-browser and pretty UI.
	
		\subsection{Summary of each members contribution}
		
				We have tried to establish fair contribution by everyone by assigning to everyone tasks on Trello, and believe that it has worked well and ensured a rather fair individual contribution. As stated beforehand, we had a clear distinction between the people working on the frontend and backend.
		
		  \subsubsection{Maciek Albin}
		    \begin{tabular}{l | p{10cm} r}
		     \emph{\large Date} & \emph{\large Comments} & \emph{\large Hours}\\
		     \hline
		     12/10/2011 & Advanced network generation for follower/folowee relations. & 3\\
		     20/10/2011 & More work on the network generation. & 4\\
		     28/10/2011 & Initial internal API definition and basic implementation. & 2\\
		     28/10/2011 & Fixes to inter thread communication. & 2\\
		     02/11/2011 & Refactoring. & 3\\
		     02/11/2011 & Implemented keyword generation using Alchemy API. & 2\\
		     03/11/2011 & Implemented basic analysis thread. & 4\\
		     04/11/2011 & Assigning tweets to stories in analysis thread. & 3\\
		     04/11/2011 & Added short summaries and links to stories DB. & 2\\
		     08/11/2011 & Added wordclouds to the API. & 1\\
		     10/11/2011 & Refactoring. & 3\\
		     30/11/2011 & Fixed bugs in the logger. & 2
		    \end{tabular}

		  \subsubsection{Jamal Khan}
		  \begin{tabular}{l | p{10cm} r}
	     \emph{\large Date} & \emph{\large Comments} & \emph{\large Hours}\\
	     \hline
		  11/10/2011 & Initial creation of the RSS Fetcher using python module feedparser. & 2\\
	    31/10/2011 & Keyword extraction completed for each story using Alchemy API. & 4\\
	    4/11/2011 & Implemented Flask Server creating API for the front end for the site as well as relevant database calls to fetch stories. & 7\\
	    10/11/2011 & Added functionality of getting stories by timestamp for the API. & 2\\
	    25/11/2011 & Completed flexible logger for each thread, so we can see exactly what is going on. & 7\\
	    12/11/2011 & Unit testing for RSS Fetcher completed. & 4
	  \end{tabular}

	  \subsubsection{Suhaib Sarmad}
	    \begin{tabular}{l | p{10cm} r}
	     \emph{\large Date} & \emph{\large Comments} & \emph{\large Hours}\\
	     \hline
	     27/10/2011 & Resource research and UI mockup & 4\\
	     28/10/2011 & Basic grid UI proof of concept & 3\\
	     09/11/2011 & Implementation of basic jQuery Masonry (tiled) UI & 2\\
	     10/11/2011 & Create tiles in UI from server news API & 4\\
	     11/11/2011 & CSS/HTML Tile layout, split into title/summary, picture, sentiment, word cloud, tweet list & 5\\
	     11/11/2011 & UI debugging and cross-browser compatibility: works on all tested browsers except IE & 2\\
	     12/11/2011 & Added tweets to tweet list using Twitter REST API and keywords from server API, added custom jQuery scrollbars & 1\\
	     12/11/2011 & Added pictures to tiles using Google Images API and article titles & 2\\
	     12/11/2011 & Generating html5 word cloud from keywords from server API & 3\\
	     19/11/2011 & CSS3 transitions research and testing for big picture UI & 2\\
	     01/12/2011 & Automatic refreshing and fetching of new news articles from server & 4\\
	     09/12/2011 & Sentiment bar prototype & 1
	    \end{tabular}


		  \subsubsection{Rafał Szymański}
		    \begin{tabular}{l | p{10cm} r}
		     \emph{\large Date} & \emph{\large Comments} & \emph{\large Hours}\\
		     \hline
		     10/10/2011 & Initial commit for the project. Basic implementation in Python of a script that given a keyword  fetches the live streaming API for the given keyword and writes it to the Mongo database. & 6\\
	       11/10/2011 & Bug fixes and a short readme. & 2\\
	       12/10/2011 & Fetching the follower/folowee graph and saving to a file. Basic analysis with NetworkX. & 3\\
	       20/10/2011 & More working on generating the user graph. & 2\\
	       21/10/2011 & Big update. Gets headlines from Google news RSS, generates keywords using Termtopia, sets up twitter stream, saves to Mongo and repeats every 5 minutes. & 7\\
	       01/11/2011 & Improving the threading of the program - added condition variables and fixed previous threading bugs. & 4\\
	       04/11/2011 & Working on instant deployment environment using Flask, Nginx, uwsgi, and a couple other technologies. Took forever to do but afterwards, after every git push, the new version is live straight away. & 10\\
	       07/11/2011 & Improvements to the instant deployment - new git hook, and a control script that allows remote restarting and clearing of the database. & 4\\
	       08/11/2011 & Wordcloud generation. Goes through each tweet, finds the most occurring words and adds that data for each analysis period. & 4\\
	       11/11/2011 & Improvements to control script. & 2\\
	       30/11/2011 & Sentiment analysis for each period. Gets sentiment using an API and adds to Mongo. & 5\\
	       04/12/2011 & Top tweets based on the number of retweets. & 2\\
	       04/12/2011 & Added all my new data to the frontend API. & 2
		    \end{tabular}

		  \subsubsection{Sam Wong}
		  \begin{tabular}{l | p{10cm} r}
	     \emph{\large Date} & \emph{\large Comments} & \emph{\large Hours}\\
	     \hline
		   12/10/2011 & Python development & 1\\
	     02/11/2011 & UI Prototype 1 & 3\\
	     09/11/2011 & UI Prototype 2 & 4\\
	     20/11/2011 & UI Prototype 3 & 5\\
	     21/11/2011 & UI debugging & 1\\
	     25/11/2011 & UI features experimentation & 1\\
	     01/12/2011 & UI features refinement & 1\\
	     01/12/2011 & Auto-refresh & 2\\
	     01/12/2011 & infinite scroll proof of concept & 3\\
	    \end{tabular}
	
	\section{Validation and Conclusions}
	
	\section{Bibliography and Tools used}
	
	Taking into account the fact that this wasn't a research project requiring extensive external material, we do not provide a full biography, but provide links to sources we have used and consulted when working on this software engineering project.
	
	\begin{itemize}
		\item Backend Resources
		\begin{itemize}
			\item 	\url{http://pyunit.sourceforge.net/} - Unit testing
			\item 	\url{http://nginx.org/} - Backend server
			\item 	\url{http://projects.unbit.it/uwsgi/} - For nginx to speak with python
			\item 	\url{http://alchemyapi.com} - API for keyword extraction
			\item 	\url{http://mongodb.com} - Data store used 
			\item 	\url{http://www.logilab.org/857} - Pylint
		\end{itemize}
		
		
		\item Frontend Resources
		\begin{itemize}
			\item 	\url{http://validator.w3.org/} - HTML Validator
			\item 	\url{http://jquery.com/} - jQuery
			\item 	\url{http://masonry.desandro.com/} - jQuery Masonry Plugin
		\end{itemize}
		
		\item Other Resources
		\begin{itemize}
			\item 	\url{http://trello.com} - Online task board
			\item 	\url{http://www.python.org/doc/} - Python documentation
			\item 	\url{http://git-scm.com/} - Version control system used
			\item 	\url{https://github.com/} - For storing code + wiki
			\item 	\url{http://flask.pocoo.org/} - Python web framework
			\item 	\url{http://news.google.com/} - News source
			\item 	Programming Collective Intelligence, Toby Segaran, OReilly Media, 2007
			\item 	Mining the Social Web, Matthew A. Russell, OReilly Media, 2011
		\end{itemize}
		
	\end{itemize} 
	
	\section{Appendix}
	
	\subsection{Team Meetings}
	\begin{tabular}{c | l p{7cm} r}
    \emph{\large Date} &  \emph{\large Venue} &  \emph{\large Subject} &  \emph{\large Attended}\\
    \hline
    14/10/2011 & Lab Round Table & Choose which project to do & \(G\)\\
    18/10/2011 & Skypeland & Choose which project to do & \(G\)\\
    21/10/2011 & Lab Round Table & Discovered the original plan is infeasible due to constraints set by Twitter, draft backup plan & \(G\)\\
    27/10/2011 & Lab Round Table & Plan approved, Design Architecture & \(G \smallsetminus \{\texttt{Suhaib}\}\)\\
    3/11/2011 & Lab Round Table & Progress Report (Backend, and UI Prototype I) & \(G \smallsetminus \{\texttt{Rafal}\}\)\\
    10/11/2011 & Lab Round Table & UI Prototype II presentation, API requests and design & \(G \smallsetminus \{\texttt{Maciej}\}\)\\
    17/11/2011 & Lab Round Table & UI Prototype III. Post-JB-meeting discussion & \(G \smallsetminus \{\texttt{Jamal}\}\)\\
    24/11/2011 & Skypeland & Incremental improvements & \(G \smallsetminus \{\texttt{Sam}\}\)\\
    1/12/2011 & Skypeland & Extensions ideas & \(G \smallsetminus \{\texttt{Suhaib}\}\)\\
    8/12/2011 & Skypeland & Features prioritisation & \(G\)\\
  \end{tabular}

  \(G=\{\texttt{Rafal}, \texttt{Sam}, \texttt{Suhaib}, \texttt{Jamal}, \texttt{Maciek}\}\)\\


	

\end{document}
